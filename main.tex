\pdfminorversion=4
\documentclass[aspectratio=169]{beamer}

\mode<presentation>
{
  \usetheme{default}
  \usecolortheme{default}
  \usefonttheme{default}
  \setbeamertemplate{navigation symbols}{}
  \setbeamertemplate{caption}[numbered]
  \setbeamertemplate{footline}[frame number]  % or "page number"
  \setbeamercolor{frametitle}{fg=white}
  \setbeamercolor{footline}{fg=black}
}

\usepackage[english]{babel}
\usepackage[utf8x]{inputenc}
\usepackage{tikz}
\usepackage{courier}
\usepackage{array}
\usepackage{bold-extra}
\usepackage{minted}
\usepackage[thicklines]{cancel}
\usepackage{fancyvrb}
\graphicspath{{figures/}}

\xdefinecolor{dianablue}{rgb}{0.18,0.24,0.31}
\xdefinecolor{darkblue}{rgb}{0.1,0.1,0.7}
\xdefinecolor{darkgreen}{rgb}{0,0.5,0}
\xdefinecolor{darkgrey}{rgb}{0.35,0.35,0.35}
\xdefinecolor{darkorange}{rgb}{0.8,0.5,0}
\xdefinecolor{darkred}{rgb}{0.7,0,0}
\definecolor{darkgreen}{rgb}{0,0.6,0}
\definecolor{mauve}{rgb}{0.58,0,0.82}
\xdefinecolor{lightyellow}{rgb}{1.0,1.0,0.75}

\title[June 29th, 2022]{%
The modern Python analysis ecosystem for high energy physics%
}
\author{Jim Pivarski, Matthew Feickert, Gordon Watts}
\institute{Princeton University, University of Wisconsin-Madison, University of Washington}
\date{June 29, 2022}

\usetikzlibrary{shapes.callouts}

\begin{document}

\logo{\pgfputat{\pgfxy(0.11, 7.4)}{\pgfbox[right,base]{\tikz{\filldraw[fill=dianablue, draw=none] (0 cm, 0 cm) rectangle (50 cm, 1 cm);}\mbox{\hspace{-8 cm}\includegraphics[height=1 cm]{princeton-logo-long.pdf}\hspace{0.1 cm}\raisebox{0.1 cm}{\includegraphics[height=0.8 cm]{iris-hep-logo-long.pdf}\hspace{2 cm}}\hspace{0.1 cm}}}}}

\begin{frame}
  \titlepage
\end{frame}

\logo{\pgfputat{\pgfxy(0.11, 7.4)}{\pgfbox[right,base]{\tikz{\filldraw[fill=dianablue, draw=none] (0 cm, 0 cm) rectangle (50 cm, 1 cm);}\mbox{\hspace{-8 cm}}}}}

% Uncomment these lines for an automatically generated outline.
%\begin{frame}{Outline}
%  \tableofcontents
%\end{frame}

% START

\begin{frame}{\mbox{ }}
\vspace{0.5 cm}
\begin{columns}
\column{0.7\linewidth}
This is a talk about measuring {\it physicists}: what they talk about and what they do for computing.

\vspace{0.25 cm}
\uncover<2->{Well-defined metrics around software are a clear objective of the NSF OAC SI2/CSSI program: we've taken that very seriously, both to try to gauge our impact and guide our evolving plans.}

\vspace{0.25 cm}
\uncover<3->{Before joining DIANA/HEP, I went from physics to data science, and had to get used to the idea of measuring people. It's a different kind of analysis: human events are {\it not} independent and systematic errors dominate.}

\vspace{0.25 cm}
\uncover<4->{Nevertheless, these kinds of analyses are meaningful: social scientists do it all the time. Inspiration: read Sharon Traweek's anthropological study of physicists at SLAC and KEK in the 1970's. Physicists can be data points!}

\column{0.3\linewidth}
\uncover<4->{\includegraphics[width=\linewidth]{iris-hep-logo-long.png}}
\end{columns}
\end{frame}

\begin{frame}{Conclusions}
\Large
\vspace{0.5 cm}
\begin{itemize}\setlength{\itemsep}{0.5 cm}
\item Different ways of understanding people, including the HEP software community: focus groups, interviews, historical documents, surveys, and proxy metrics.

\item This talk focused on proxy metrics, which are quantitative, but you have to pay close attention to what they're quantifying.

\item Some clear trends and conclusions emerged. \mbox{Others are muddled.\hspace{-1 cm}}
\end{itemize}
\end{frame}

\end{document}
